\section{Chaos as a concept}
\vspace{0.3in}
\begin{center}
    \textbf{Chaos}

    \textit{"A state of total confusion with no order."}
\end{center}

This is the definition of Chaos given by the Cambridge Dictionary, and
a perfect description of the initial impressions one has observing such a system.\\
Yet physical systems like the double pendulum, vibrating objects, 
rotating or heated fluids, the motion of a group of celestial objects and dripping faucets
are deterministic in nature and can be derived from Newtons laws \cite{baker}.\\
Why is it then that we describe them as "total confusion"? To answer this question we look at
the definition to a Chaotic System given by Devaney \cite{devaney}:

\begin{center}
    Let $X$ be a metric space. A continuous map $f:\; X\to X$ is said to be
    chaotic on $X$ if:
\end{center}   

\begin{enumerate}
    \item $f$ is transitive,
    \item the periodic points of $f$ are dense in $X$,
    \item $f$ has sensitive dependence on initial conditions.
\end{enumerate}

Each of the three conditions is assured by the existences of the other two \cite{banks}, \cite{crannell}.\\
All three are listed for historical reasons and for the strength of the definition, but 
for practical purposes if two are demonstrated to be valid for a system we will
consider it chaotic.\\
The last condition is perhaps most intuitive. Imagine two systems, one of which starting with
slightly different initial conditions to the other. For non-chaotic systems this difference will 
grow linearly with time. The difference between chaotic systems, however, grows \textit{exponentially}
leading to dissimilar behavior very quickly. \\
Elaborating on the $1^{st}$ condition. A continuous map or in our case, a dynamical system
$f:\; X\to X$ is said to be transitive if, for any pair $U, V$ of nonempty open subsets of $X$
there exists some positive number of applications $n\geq 0$ of the map $f$ such that
$f^n(U)\cap V \neq \emptyset$. \cite{erdmann} \\
In Layman's terms the above claims that through some $n$ number of applications of $f$, defined hereafter as $f^n$, we can reach any
point belonging to $X$. The $2^{nd}$ rule states that every reachable point belongs to $X$.
\paragraph{}
Chaotic systems can be generalised into two families based on the equations that describe them:\\
1. Linear\\
2. Non-linear\\
In our day to day life we predominantly encounter non-linear chaos. Systems being influenced by a myriad of unpredictable
and untrackable circumstances. In this work we will study the first kind, which will require us to observe the behaviour
of the systems in the long term. 

    
